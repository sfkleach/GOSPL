\documentstyle[11pt]{article}

\title{\bf The Grobbit (or ``Whiffle while you work'')}

\author{\bf Chris Price}

\parskip=6pt plus 2pt minus 3pt
\parindent=0pt
\topsep=0pt

\hyphenpenalty=8000
\doublehyphendemerits=10000
\finalhyphendemerits=10000
\begin{document}
\maketitle

\section{Your task}

You are in charge of a family of grobbits --- grobbits are small, harmless,
furry creatures living in the land of Tinyglobe. Tinyglobe looks quite
similar to an N by N grid where the edges of the grid wrap around.

The task before you is to evolve a strategy to keep your grobbits safe
and well fed in a world that is fairly hostile. You are provided with a
default strategy where your grobbit will give you a report on what is
happening and prompt you to make a decision each turn. You can use this
to try out being a grobbit. However, your task is to specify a strategy
for the grobbits which will allow them to stand on their own two feet
(metaphorically speaking --- grobbits actually walk on all fours).

\section{Implementation Details}

Both the grobbit and the other creatures are implemented as "objects"
in Tinyglobe, where each object is at a particular square on the N by N
grid. Each object takes it in turn to do what it wants, so after the grobbit
has moved then all its enemies get a turn. If during an enemy's turn
it manages to occupy the same square, it can choose to interact
(i.e. to eat you).

Other items in the world are: trees that can be climbed; holes that can
be hidden in; wiffle flowers that grow in random locations and are a
grobbit's favourite food.

Grobbits have certain abilities that you are not allowed to change ---
the speed at which they can move, the distance that they can see, the
frequency with which they need food. Neither are you allowed to change
the properties of other items in the world apart from grobbits. What
you are allowed to change is the cunning with which grobbits survive
in their world.

How do you do that? You implement a better version of the method
"next\_turn". It can be as big as you like (although it might be
better to do your implementation as a series of procedure calls rather
than one great big method). It will need to invoke other methods, and the
following section gives details of what you are allowed to change ---
anything else is cheating and will be frowned upon.

\section{Practical Details}

There are three types of routines implemented in the file {\em world.p} that
you are allowed to call from your version of next\_turn:

\begin{description}

\item [Grobbit methods and ivars.]

\begin{enumerate}

\item  \verb+^+move( position ). This will move you to the new position
(expressed as a list with two elements), assuming that the new position is
within your range. You can make more than one move per turn, but only for
\verb+^+speed squares.

\item \verb+^+speed. This returns the number of squares that you can move per
turn (not necessarily in a straight line, but a diagonal move counts as two
squares).

\item \verb+^+look. This returns a list containing references to all the
objects within your visibility range. You can then send messages to those
enemies to find out about them.

\item \verb+^+up\_a\_tree, \verb+^+down\_a\_hole. True if you are in that
particular position. In either case, you need to get out of it before you
move.

\item \verb+^+climb\_a\_tree, hide\_in\_hole, ground\_level. Get in a tree or a
hole, or get out of it again. To climb a tree or go down a hole, you must be
on the same square as it.

\item \verb+^+eat\_food( object). Object should be something edible at the
same location (e.g. a wiffle flower). If not, then you can't eat it. 

\item \verb+^+nourishment. How many turns you can survive without finding
something else edible. 

\item \verb+^+position. Your current location.

\end{enumerate}

\item  [Other object methods and ivars.]

\begin{enumerate}

\item  creature\_object \verb+<-+ speed. Returns how many squares enemy can move
next turn. 

\item tree\_object \verb+<-+ are\_you\_empty. Can only climb an empty one.

\item hole\_object \verb+<-+ are\_you\_empty. Can only go down an empty one.

\item creature\_object \verb+<-+ get\_name. Returns the name of the class that the
creature belongs to. 

\item creature\_object \verb+<-+ potential\_food. Returns a list of the objects that
the creature likes to eat. If member( ``grobbit'', creature\_object \verb+<-+
potential\_food), then the creature might want to eat you.

\item any\_object \verb+<-+ position. Returns where exactly on the board the
creature is. 

\end{enumerate}

\item [Simple routines.]

\begin{enumerate}

\item lpr, nlpr, nls. Simple routines for printing lists and blank lines. 

\item distance\_away( position1, position2 ). Returns how many
squares apart the two positions are (useful when calculating threats - how
far away and how fast is that guy?).

\end{enumerate}

\end{description}


A basic template for you to start altering is provided in {\em grobbit.p}.
{\em world.p} and {\em test1.datafile} are the rest of the Tinyglobe
code and a test run respectively.

Type {\bf tryout} to the Pop interpreter in order to run the grobbit world.

\end{document}
